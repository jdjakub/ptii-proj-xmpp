\chapter{Preparation}
For the bulk of the preparatory work, the clue was in the title: I needed to be familiar with OCaml and XMPP. This involved maintaining a basic level of competence with OCaml language features, as well as experience with the libraries and tools I was going to use; understanding of the basic concepts of XML, and which features are relevant to XMPP; then, the details of the XMPP protocol itself. Once I had a vision of what would be needed, I could design the general architecture of the server.

\section{Starting point}

\section{The OCaml language}
\begin{itemize}
  \item Statically typed, strong type system, functional with imperative features, object-oriented bits, small standard library
  \item Already used it over summer with Angstrom, but at a basic level.
  \item Unix networking functions
  \item POSIX Threads library, concurrency
  \item Basic cl/sv demos
  \item Compilation, makefile
\end{itemize}

Objective-Caml, or OCaml, is a primarily functional language with imperative and object-oriented features. I had already used OCaml, to a basic extent, on an internship during the summer; this had given me some experience with the Angstrom parser combinator library. I referred to Internet resources and the \name{Real World OCaml} book to re-familiarise myself and fill the gaps in my knowledge.

I surveyed the documentation for the \code{Unix} module, to remind myself of how the Sockets API operates. I put together a couple of small test servers and clients to verify I could use the networking functions from OCaml correctly. This also provided an opportunity to get used to the build system again.

I also browsed the small standard library for data structure implementations. All of the basic ones exist, and more specialised versions are provided in external libraries should I need them.

The server needed to handle many clients at once, so I needed some form of multitasking. OCaml offers two co-operative threading libraries, \code{Lwt} and \code{Async}, in addition to an interface to POSIX pre-emptive threads. When I later had to adapt my single-threaded code to serve multiple clients, I chose POSIX threads, since the pre-emptive multithreading approach was already familiar to me and I did not need to make serious modifications to the code I had already written.

\section{XML}
\begin{itemize}
  \item Tree-structured data
  \item spork from impl
  \item language
  \item encoding, schemas, validation, comments, ...
  \item grammar
\end{itemize}

XML is a notation for hierarchical trees, and an XML node \footnote{Technically, there are also text nodes: as XML was designed as a markup language, any point in the tree can be made up of text, or `content'.} has the form
\begin{lstlisting}[language=xml]
<prefix:tag pre1:attr1="value1" pre2:attr2="value2" ... pren:attrn="valuen" >
   ... children ...
</prefix:tag>
\end{lstlisting}
where \code{attr}\(i\) are the node's attributes. Most of the time, prefixes are absent from attributes, but they are used in a couple of important cases outlined below. I call the combination of prefix and identifier a \emph{qualified name}. This way, a node can be seen as
\begin{lstlisting}[language=xml]
<qname qname1="value1" qname2="value2" ... qnamen="valuen" >
  ...
</qname>
\end{lstlisting}

Namespaces are a way to organise tags and avoid naming conflicts in XML; each tag is qualified by a namespace. Namespaces are typically long strings, and it would be cumbersome to work with XML where every single tag had such a namespace concatenated onto it. Instead, a shorthand for the namespace is prefixed onto tags. The association between a prefix and the namespace it represents is defined by an (ab)use of the XML attribute system: the `attribute' \code{xmlns:foo="bar"} is treated specially as saying that the prefix \code{foo} represents the namespace \code{bar}. A tag can in fact forego a prefix, in which case it is considered to use the default prefix, signified by \code{xmlns="bar"}. These associations are local to the node in the tree and are inherited by its children.

There are other special `attributes' in this fashion, such as \code{xml:lang}, which specifies the language used for plain text within the subtree it is part of.

\section{The XMPP Protocol}
\begin{itemize}
  \item sources: RFC, book, Psi
  \item strategy: breadth-first not depth-first
  \item XML streams; errors
  \item The handshake; authentication, encryption ...
  \item roster (data structs; spork from impl)
  \item Messaging and stanzas (data structs; threading)
\end{itemize}

\section{Requirements analysis}

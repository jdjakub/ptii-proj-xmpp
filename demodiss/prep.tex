\chapter{Preparation}
For the bulk of the preparatory work, the clue was in the title: I needed to be familiar with OCaml and XMPP. This involved maintaining a basic level of competence with OCaml language features, as well as experience with the libraries and tools I was going to use; understanding of the basic concepts of XML, and which features are relevant to XMPP; then, the details of the XMPP protocol itself. Once I had a vision of what would be needed, I could design the general architecture of the server.

\section{Starting point}
I relied on the existing OCaml platform, including its standard libraries and others such as Angstrom.

\section{The OCaml language}
Objective-Caml, or OCaml, is a primarily functional language with imperative and object-oriented features. I had already used OCaml, to a basic extent, on an internship during the summer; this had given me some experience with the Angstrom parser combinator library. I referred to Internet resources and the \name{Real World OCaml} book to re-familiarise myself and fill the gaps in my knowledge.

I surveyed the documentation for the \code{Unix} module, to remind myself of how the Sockets API operates. I put together a couple of small test servers and clients to verify I could use the networking functions from OCaml correctly. This also provided an opportunity to get used to the build system again.

I also browsed the small standard library for data structure implementations. All of the basic ones exist, and more specialised versions are provided in external libraries should I need them.

The server needed to handle many clients at once, so I needed some form of multitasking. OCaml offers two co-operative threading libraries, \code{Lwt} and \code{Async}, in addition to an interface to POSIX pre-emptive threads. When I later had to adapt my single-threaded code to serve multiple clients, I chose POSIX threads, since the pre-emptive multithreading approach was already familiar to me and I did not need to make serious modifications to the code I had already written.

\section{XML}
XML is a notation for hierarchical trees, and an XML node \footnote{Technically, there are also text nodes: as XML was designed as a markup language, any point in the tree can be made up of text, or `content'.} has the form
\begin{xml}
<prefix:tag pre1:attr1="value1" pre2:attr2="value2" ... pren:attrn="valuen" >
   ... children ...
</prefix:tag>
\end{xml}
where \code{attr}\(i\) are the node's attributes. Most of the time, prefixes are absent from attributes, but they are used in a couple of important cases outlined below. I call the combination of prefix and identifier a \emph{qualified name}. This way, a node can be seen as
\begin{xml}
<qname qname1="value1" qname2="value2" ... qnamen="valuen" >
  ...
</qname>
\end{xml}

\subsection{Namespaces}
Namespaces are a way to organise tags and avoid naming conflicts in XML; each tag is qualified by a namespace. Namespaces are typically long strings, and it would be cumbersome to work with XML where every single tag had such a namespace concatenated onto it. Instead, a shorthand for the namespace is prefixed onto tags. The association between a prefix and the namespace it represents is defined by an (ab)use of the XML attribute system: the `attribute' \code{xmlns:foo="bar"} is treated specially as saying that the prefix \code{foo} represents the namespace \code{bar}. A tag can in fact forego a prefix, in which case it is considered to use the default prefix, signified by \code{xmlns="bar"}. These associations are local to the node in the tree and are inherited by its children.

There are other special `attributes' in this fashion, such as \code{xml:lang}, which specifies the language used for plain text within the subtree it is part of. I deemed the possibly many other special attributes not relevant to this project. Nor did I concern myself with other beauraucratic details such as file encoding, validation via XML schemas or comments / \code{CDATA} sections.

\subsection{Formal grammar for XML}
I developed the following informal Context-Free Grammar for this subset of XML, to guide my implementation of the parser:

\begin{lstlisting}
qname    ::= (ident ':')? ident
attr_val ::= qname '=' '"' string '"'
branch   ::= '<' qname attr_val* tag_end
tag_end  ::= '/>'
           | '>' node '</' qname '>'

node     ::= branch
           | text

\end{lstlisting}
In the CFG, \code{ident} denotes an identifier, \code{string} represents a text string containing appropriately escaped quote characters, and \code{text} refers to free text not containing angle bracket characters. This was strictly a guide, to give the general shape of the syntax. Obviously XML parsing is not \emph{truly} context-free as the closing tag must match the opening tag; nevertheless, such a grammar is still useful.

\section{The XMPP Protocol}
\begin{itemize}
  \item sources: RFC, book, Psi
  \item strategy: breadth-first not depth-first
  \item XML streams; errors
  \item The handshake; authentication, encryption ...
  \item roster (data structs; spork from impl)
  \item Messaging and stanzas (data structs; threading)
\end{itemize}
\subsection{Understanding the protocol}
The most comprehensive and in-depth resource on XMPP is the specification, RFC 6120. My first attempt at understanding XMPP involved trying to read this document. Since at first all I wanted was a broad picture, I found the book \name{XMPP: The Definitive Guide} to be a more appropriate resource.

Once I had a broad overarching understanding, the only place I could start with was the communications setup and handshake. The \name{Definitive Guide} was helpful, but for the details I had to delve into the RFC. I quickly realised that XMPP is a very large and complex protocol, and I would have to prioritise the aspects that would get me to a message-processing stage as fast as possible. In other words, I needed to adopt a breadth-first, and not depth-first, approach to both understanding XMPP and implementing a server for it. Thus, for each feature of the protocol, I pressed ahead with the core behaviour first. The specification contains a rather overwhelming range of error cases and caveats for most aspects of XMPP, and I would treat these details as extension work for any remaining time at the end.

To help guide me through the relevant sections of RFC 6120, I downloaded the XMPP client program Psi\cite{Psi-IM} and learned its basic use. I would treat Psi as a typical client and make the server conform to the data it sent.

What follows in subsequent sections is my consideration of specific parts of XMPP, and the design decisions affecting their representation in my server.

\subsection{Handshake and setup considerations}

\subsection{The roster}
`Roster' is simply XMPP jargon for what is functionally a contact list. XMPP represents a user's contacts as a list of \code{<item>} elements. Each item describes another user, and the subscription status between the two users.

The concept of subscription has design ramifications worth discussing. The word is used in its usual sense: if user A is subscribed to user B, then A will receive updates about B's presence on the network. This is a \emph{directed} relation; B might not be subscribed to A. Thus, conceptually, subscription is a directed graph that needs to be stored on disk as well as in memory. There is, then, a choice of how to represent this graph.

The two main options for graph representation are \emph{adjacency list} and \emph{adjacency matrix}.

An adjacency matrix, being a conceptually two-dimensional structure, would have forced me to think about implementation in terms of nested one-dimensional maps or arrays. This would complicate the issue of avoiding loading the entire graph at once, and hence how to split it into separate files or regions of a file, and so on. On the other hand, an adjacency list for each node would be simple to implement using conventional files and in-memory data structures. Furthermore, this already matches up to the way XMPP treats the subscription relation; a list of contacts per user. When the user connects, simply load the list.

Thus, I picked the adjacency list. I would have a separate file for each `vertex' of the graph, in a format resembling a list of \code{<item>}s, and similar separate in-memory data structures, making loading and lookup simple to think about.

For example, the username \code{alice} has a roster file located in the \code{roster/} directory called \code{alice.xml}. It consists of a list of XMPP roster \code{<item>} elements.

The subscribers of user A are recorded in the XMPP \code{<item>} elements as \code{from}, \code{to}, or \code{both}. RFC 6121 defines \code{to} as A having a subscription to the user in the \code{<item>} element; \code{from} is the converse, and \code{both} simply means A and B are mutually subscribed.

Now, if A's roster (on disk or in memory) says it is subscribed \code{to} B, then B's roster should say it is subscribed \code{from} A. The analogous case applies to \code{from}/\code{to}. This is an invariant that must be maintained if the roster, taken as a whole, is to be consistent, and is a consequence of the adjacency list representation.

[but --- no need to enforce, since I abandoned CRUD operations and can trust my own files to be consistent]
However, I viewed supporting CRUD (Create, Read, Update, Delete) operations for the roster as extension work, since implementing it before messaging would be too time consuming. The rosters are determined by the content of the roster files which I ensure are consistent myself, and since they do not change during runtime or get updated by the server, this consistency is maintained.

\subsection{Messaging and stanzas}

\subsection{Multi-client architecture}

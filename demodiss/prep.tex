\chapter{Preparation}
For the bulk of the preparatory work, the clue was in the title: I needed to be familiar with OCaml and XMPP. This involved maintaining a basic level of competence with OCaml language features, as well as experience with the libraries and tools I was going to use; understanding of the basic concepts of XML, and which features are relevant to XMPP; then, the details of the XMPP protocol itself. Once I had a vision of what would be needed, I could design the general architecture of the server.

\section{Starting point}
I relied on the existing OCaml platform, including its standard libraries and others such as Angstrom.

\section{The OCaml language}
Objective-Caml, or OCaml, is a primarily functional language with imperative and object-oriented features. I had already used OCaml, to a basic extent, on an internship during the summer; this had given me some experience with the Angstrom parser combinator library. I referred to Internet resources and the \name{Real World OCaml} book to re-familiarise myself and fill the gaps in my knowledge.

I surveyed the documentation for the \code{Unix} module, to remind myself of how the Sockets API operates. I then put together a couple of small test servers and clients to verify I could use the networking functions from OCaml correctly. This also provided an opportunity to get used to the build system again.

I also browsed the small standard library for data structure implementations. All of the basic ones exist, and more specialised versions are provided in external libraries should I need them.

The server needed to handle many clients at once, so I needed some form of multitasking. OCaml offers two co-operative threading libraries, \code{Lwt} and \code{Async}, in addition to an interface to POSIX pre-emptive threads. When I later had to adapt my single-threaded code to serve multiple clients, I chose POSIX threads, since the pre-emptive multithreading approach was already familiar to me and I did not need to make serious modifications to the code I had already written.

\section{XML}
XML is a notation for hierarchical trees, and an XML node \footnote{Technically, there are also text nodes: as XML was designed as a markup language, any point in the tree can be made up of text, or `content'.} has the form
\begin{xml}
<prefix:tag pre1:attr1="value1" pre2:attr2="value2" ... pren:attrn="valuen" >
   ... children ...
</prefix:tag>
\end{xml}
where \code{attr}\(i\) are the node's attributes. Most of the time, prefixes are absent from attributes, but they are used in a couple of important cases outlined below. I call the combination of prefix and identifier a \emph{qualified name}. This way, a node can be seen as
\begin{xml}
<qname qname1="value1" qname2="value2" ... qnamen="valuen" >
  ...
</qname>
\end{xml}

\subsection{Namespaces}\label{sec:namespaces}
Namespaces are a way to organise tags and avoid naming conflicts in XML; each tag is qualified by a namespace. Namespaces are typically long strings, and it would be cumbersome to work with XML where every single tag had such a namespace concatenated onto it. Instead, a shorthand for the namespace is prefixed onto tags. The association between a prefix and the namespace it represents is defined by an (ab)use of the XML attribute system: the `attribute' \code{xmlns:foo="bar"} is treated specially as saying that the prefix \code{foo} represents the namespace \code{bar}. A tag can in fact forego a prefix, in which case it is considered to use the default prefix, signified by \code{xmlns="bar"}. These associations are local to the node in the tree and are inherited by its children.

There are other special `attributes' in this fashion, such as \code{xml:lang}, which specifies the language used for plain text within the subtree it is part of. I deemed the possibly many other special attributes not relevant to this project, and did not investigate them. Nor did I concern myself with other beauraucratic details such as file encoding, validation via XML schemas or comments / \code{CDATA} sections.

\subsection{Formal grammar for XML}
I developed the following informal Context-Free Grammar for this subset of XML, to guide my implementation of the parser:

\begin{minipage}{\linewidth}
  \begin{lstlisting}
  qname    ::= (ident ':')? ident
  attr_val ::= qname '=' '"' string '"'
  branch   ::= '<' qname attr_val* tag_end
  tag_end  ::= '/>'
             | '>' node '</' qname '>'

  node  ::= branch
         | text

  \end{lstlisting}
\end{minipage}

In the CFG, \code{ident} denotes an identifier, \code{string} represents a text string containing appropriately escaped quote characters, and \code{text} refers to free text not containing angle bracket characters. This was strictly a guide, to give the general shape of the syntax. Obviously XML parsing is not \emph{truly} context-free as the closing tag must match the opening tag; nevertheless, such a grammar is still useful.

\section{The XMPP Protocol}

\subsection{Understanding the protocol}
The most comprehensive and in-depth resource on XMPP is the core specification, RFC 6120. My first attempt at understanding XMPP involved trying to read this document. Since at first all I wanted was a broad picture, I found the book \name{XMPP: The Definitive Guide} to be a more appropriate resource.

Once I had a broad overarching understanding, the only place I could start with was the communications setup and handshake. The \name{Definitive Guide} was helpful, but for the details I had to delve into the RFC. I quickly realised that XMPP is a very large and complex protocol, and I would have to prioritise the aspects that would get me to a message-processing stage (detailed in XMPP-IM, RFC 6121) as fast as possible. In other words, I needed to adopt a breadth-first, and not depth-first, approach to both understanding XMPP and implementing a server for it. Thus, for each feature of the protocol, I pressed ahead with the core behaviour first. The specification contains a rather overwhelming range of error cases and caveats for most aspects of XMPP, and I treated these details as extension work for any remaining time at the end.

To help guide me through the relevant sections of RFC 6120 and 6121, I downloaded the XMPP client program Psi\cite{Psi-IM} and learned its basic use. I would treat Psi as a typical client and make the server conform to the data it sent.

What follows in subsequent sections is my consideration of specific parts of XMPP, and the design decisions affecting their representation in my server.

\subsection{Handshake and setup considerations}
Communication in XMPP revolves around sending stanzas over XML streams. After TCP connection, the client and server need to set up their streams and negotiate various features before they can begin exchanging stanzas. This is known as the handshake. I see the handshake as consisting of the following steps:
\begin{enumerate}
  \item Set up the streams. The client and server exchange \code{<stream>} opening tags containing the server name, protocol version (1.0) and a stream ID for the client to hold onto.

  \item Offer the list of stream features that the client can choose from. These are the subject of numerous extensions and concern topics such as verified authentication and encryption of the streams. The only capability \emph{required} by the specification, and by the Psi client, is to have some form of authentication mechanism. The simplest way to do this is to offer authentication via the \code{PLAIN} mechanism, receiving the client's username and password in base64-encoded plain text. This can then be accepted by default. Specific user accounts and secure authentication are beyond the scope of this project.

  \item Respond with `success' to client. Restart the stream anew by handshaking again\footnote{This is mainly intended for encrypted connections and suchlike, so that the entire stream can be considered secured; nevertheless, it is still required in other cases.}. Offer features again, this time consisting of the required process of ``resource binding''.

  \emph{Resources} in XMPP allow the same user to communicate on different devices; they are appended to a client's Jabber ID, or JID\footnote{XMPP grew out of the older Jabber protocol, and users are identified by their JIDs. A JID looks very similar to an e-mail address, consisting of a username and the name of the server on which they reside: \code{username@server.net}.}, to make a \emph{full JID}. For example, the user \code{alice} could possess two full JIDs \code{alice@server.net/laptop} and \code{alice@server.net/phone}. The resource name is suggested by the client and either accepted or modified (in the case of clashes) by the server, which returns the full JID to the client. Resource binding is accomplished by \code{iq} (Information Query) stanzas.

  \item Expect an \code{iq} `set' session request, to formally establish the session, and respond with an empty \code{iq} signalling success.

  \item Expect an \code{iq} `get' request for the roster (Section~\ref{sec:prep-roster}), respond with client's roster.

  \item Expect an initial \code{presence} stanza, meaning ``client online''. Echo this back to the client, with its full JID in the \code{from} attribute, and forward to its subscribers.
\end{enumerate}

Once these steps are completed, ordinary \code{message}, \code{presence} and \code{iq} stanzas can be exchanged.

If any errors occur at this stage, the stream is terminated by sending a stream error and the closing \close{</stream>} tag. However, due to the number of varied errors that can occur, graceful termination is an unhelpful goal to prioritise and simply closing the \code{<stream>} element will suffice for my purposes.

\subsection{Messaging and stanzas}
Outline:
\begin{itemize}
  \item Sticking to simple messages with \code{to}, \code{from}, \code{type} attributes; oblivious to nested XML, simply pass along
  \item Sticking to online-offline (\code{avaliable} / \code{unavailable}) presence notification only
  \item Sticking to single-server model; i.e. no federation (``yet'')
  \item Only implementing the \code{iq} stanzas necessary for handshake; too many potential services otherwise. Graceful (spec-sanctioned) error response (\code{feature-unavailable} error stanza) so that more feature-rich clients (read: Psi) can still continue to operate with the server.
\end{itemize}

\subsection{The roster}\label{sec:prep-roster}
`Roster' is simply XMPP jargon for what is functionally a contact list. XMPP represents a user's contacts as a list of \code{<item>} elements. Each item describes another user, and the subscription status between the two users.

The concept of subscription has design ramifications worth discussing. The word is used in its usual sense: if user A is subscribed to user B, then A will receive updates about B's presence on the network. This is a \emph{directed} relation; B might not be subscribed to A. Thus, conceptually, subscription is a directed graph that needs to be stored on disk as well as in memory. There is, then, a choice of how to represent this graph.

The two main options for graph representation are \emph{adjacency list} and \emph{adjacency matrix}.

An adjacency matrix, being a conceptually two-dimensional structure, would have forced me to think about implementation in terms of nested one-dimensional maps or arrays. This would complicate the issue of avoiding loading the entire graph at once, and hence how to split it into separate files or regions of a file, and so on. On the other hand, an adjacency list for each node would be simple to implement using conventional files and in-memory data structures. Furthermore, this already matches up to the way XMPP treats the subscription relation; a list of contacts per user. When the user connects, simply load the list.

Thus, I picked the adjacency list. I would have a separate file for each `vertex' of the graph, in a format resembling a list of \code{<item>}s, and similar separate in-memory data structures, making loading and lookup simple to think about.

For example, the username \code{alice} has a roster file located in the \code{roster/} directory called \code{alice.xml}. It consists of a list of XMPP roster \code{<item>} elements.

The subscribers of user A are recorded in the XMPP \code{<item>} elements as \code{from}, \code{to}, or \code{both}. RFC 6121 defines \code{to} as A having a subscription to the user in the \code{<item>} element; \code{from} is the converse, and \code{both} simply means A and B are mutually subscribed.

Now, if A's roster (on disk or in memory) says it is subscribed \code{to} B, then B's roster should say it is subscribed \code{from} A. The analogous case applies to \code{from}/\code{to}. This is an invariant that must be maintained if the roster, taken as a whole, is to be consistent, and is a consequence of the adjacency list representation.

However, I viewed supporting CRUD (Create, Read, Update, Delete) operations for the roster as extension work, since implementing them before messaging would be too time consuming. The rosters will therefore be determined by the content of the roster files, and I will manually ensure their consistency. Since they do not change during runtime or get updated by the server, this consistency is maintained.

\subsection{Multi-client architecture}
This section is supposed to explain the design behind Section \ref{sec:mod-dispatch} (dispatch machinery) as well as
\begin{itemize}
  \item Developed first for a single client; multiple clients perform the same actions specified by the same function, just in  different thread
  \item Design of the testing clients; explaining the analogous Section \ref{sec:mod-client} in Implementation, as well as the test rig
\end{itemize}

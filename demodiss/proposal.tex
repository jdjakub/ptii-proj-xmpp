% Note: this file can be compiled on its own, but is also included by
% diss.tex (using the docmute.sty package to ignore the preamble)
\documentclass[12pt,a4paper,twoside]{article}
\usepackage[pdfborder={0 0 0}]{hyperref}
\usepackage[margin=25mm]{geometry}
\usepackage{graphicx}
\usepackage{parskip}
\begin{document}

\begin{center}
\Large
Computer Science Tripos -- Part II -- Project Proposal\\[4mm]
\LARGE
An XMPP Server Implementation in OCaml\\[4mm]

\large
Joel Jakubovic, Pembroke College

Originators: Dr Anil Madhavapeddy \& Dr Richard Mortier

21 October 2016
\end{center}

\vspace{5mm}

\textbf{Project Supervisor:} Dr Richard Mortier

\textbf{Director of Studies:} Dr Austen Donnelly

\textbf{Project Overseers:} Dr Sean Holden  \& Dr Simone Teufel

% Main document

\section*{Introduction}

The eXtensible Messaging and Presence Protocol (XMPP) is designed for securely
passing XML messages between clients and servers. Amongst its services
are ordinary instant-messaging functionality, storage of contacts, discovery of
online status of users, and push-based notifications. Google and Apple employ
it for the latter use, and it is central to the WhatsApp messaging application.
There are implementations of many protocols in OCaml for MirageOS, but
not for XMPP. Therefore, this project concerns itself with using OCaml to
implement a server for XMPP.

\section*{Starting point}

I will be relying on the existing OCaml platform, including its standard
libraries and implementations of security protocols, such as TLS, as well as
concurrency libraries (Async, Lwt). I will also use the Angstrom parser
combinator library to construct parsers for protocol messages.

\section*{Resources required}

I will be using my own laptops - one Windows and one MacBook - for development
and rudimentary testing. Since this project develops server software, I will
make use of a server machine to better match the real environment the software
would run in, for performance testing.

I accept full responsibility for my machines and I have made contingency plans
to protect myself against hardware and/or software failures. I will use github
for backup. If github is not available, I will clone to my University-provided
file space. If the Internet is inaccessible, I will clone to a pen drive and
push as soon as possible.

\section*{Work to be done}

The project breaks down into the following main sub-tasks:

\begin{enumerate}

\item Learning the tools I will need. This is mainly concerned with OCaml.
  First, I will need to simultaneously learn the OCaml language and build
  system. I will also need to gain a general experience of the networking,
  threading and parsing libraries, as well as the concepts underlying them if
  I am not already familiar.

\item Understanding the XMPP protocol. Read RFCs, books and Internet resources.
  Design and prototype data structures and algorithms as this happens, as
  appropriate.

\item Implementing the parser for XMPP packets. This will involve general XML
  parsing since XMPP is XML-based, however it is not likely that the whole
  stack of XML schema validation, etc. will be required.

\item Implementing the semantics of the Core protocol; messages, authentication
  and presence.

\item Testing the implementation for correctness and running performance
  benchmarks.

\item Writing the dissertation documenting the project. I will need to learn and
  practise LaTeX for this.

\end{enumerate}

\section*{Success criteria}

The project will be deemed a success if the following two conditions are met:

\begin{enumerate}

\item The software conforms to the Core Compliance Requirements for XMPP servers: \texttt{https://xmpp.org/rfcs/rfc3920.html\#compliance}. This shall be demonstrated by running test suites and giving representative samples of server behaviour alongside the main points of functional specification in the RFC.

\item The software is reasonably efficient and does not have serious performance issues.
  This will involve consideration of throughput (in messages per second, and bitrate) and memory footprint (as a function of number of simultaneous connections, and XML nesting depth).

\end{enumerate}

\section*{Possible extensions}

If the main result is achieved with extra time to spare, then I will build a Mirage-compatible
implementation by performing IO through the Mirage API.

\section*{Timetable}

Planned starting date is 22/10/2016.

\begin{enumerate}

\item \textbf{22/10--04/10 (weeks 3,4)} Become comfortable with the OCaml language and build system.
  Decide which specific libraries to use in the case of alternatives (e.g. Lwt vs Async). Become familiar
  with these libraries. Client / server socket setup demo. Send / receive data demo.

\item \textbf{04/11--18/11 (weeks 5,6)} Study XMPP. Use it to inform design of data structures and algorithms.
  Implement handshaking, and the parser for XMPP stanzas. Demonstrate through small demo programs: respond
  to client handshake request; accept message from client, display abstract syntax structure of message. Use
  existing XMPP client or own programs as necessary.

\item \textbf{18/11--01/12 (weeks 7,8)} Implement semantics for ordinary messages, presence messages and
  info query (iq) messages, under a single-threaded server model (i.e. one connection only).

\item \textbf{Michaelmas vacation} Switch to multithreaded model capable of handling multiple simultaneous connections.

\item \textbf{20/01--03/02 (weeks 1,2)} Write progress report and prepare presentation.

\item \textbf{03/02-17/02 (weeks 3,4)} Evaluate existing components and do initial performance tests.
  Add roster management functionality.

\item \textbf{17/02-03/03 (weeks 5,6)} Finish implementation of core XMPP spec, test and evaluate.

\item \textbf{03/03-17/03 (weeks 7,8)} Write main chapters of dissertation.

\item \textbf{Easter vacation} Finish dissertation and proof-read. Slack time, in case of overrun --- otherwise, if time permits, extension work.

\item \textbf{by 21/04/17} First draft of dissertation to be sent to supervisor for feedback.

\item \textbf{by 05/05/17} Final draft of dissertation.

\end{enumerate}

\end{document}
